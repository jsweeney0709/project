\documentclass[oneside,openright,frontopenright]{dmathesis}
\input{preamble}


\begin{document}
\title{Null geodesics}
\subtitle{Modelling light in the Schwartzschild metric}
\author{Joseph A Sweeney}
\researchgroup{Mathematics}
\pagenumbering{roman}
\maketitlepage*

\begin{abstract}
%
	The field of black hole imaging in Physics and Computer Science has had a wave of 
	enthusiasm and development following Christopher Nolan’s 2014 Interstellar, and the 
	imaging of the supermassive black hole at the centre of M87 by the Event Horizon 
	Telescope in 2019. Producing high quality images and videos of black holes and 
	wormholes can be very computationally costly, so ingenious ways of gaining efficacy 
	must often be used. My aim is to explore these methods and tricks to present practical 
	approaches to modelling various spacetime metrics, 
	including the Schwarzschild metric, Kerr metric, and the Ellis wormhole.
%
\end{abstract}

\begin{declaration*}
%
	The work in this thesis is based on research carried out in the Department of
	Mathematical Sciences at Durham University. No part of this thesis has been
	submitted elsewhere for any degree or qualification.
%
\end{declaration*}

\disableprotrusion
\tableofcontents*
\enableprotrusion

\cleardoublepage
\pagenumbering{arabic}

\include{intro}
%\include{background}
%\include{paper1}
%\include{paper2}
%\include{paper3}
\begin{introduction}

	The field of black hole imaging in Physics and Computer Science has had a wave of enthusiasm 
	and development following Christopher Nolan’s 2014 Interstellar, and the imaging of the supermassive 
	black hole at the centre of M87 by the Event Horizon Telescope in 2019. Producing high quality images 
	and videos of black holes and wormholes can be very computationally costly, so ingenious ways of gaining 
	efficacy must often be used. My aim is to explore these methods and tricks to present practical approaches 
	to modelling various spacetime metrics, including the Schwarzschild metric, Kerr metric, and the Ellis wormhole.

\end{introduction}

\begin{chapter1}

\end{chapter1}

\appendix
%\include{appendix1}
%\include{appendix2}

\nocite{*}
\printbibliography[heading=bibintoc]

\end{document}